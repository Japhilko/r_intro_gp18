\documentclass[ignorenonframetext,]{beamer}
\setbeamertemplate{caption}[numbered]
\setbeamertemplate{caption label separator}{: }
\setbeamercolor{caption name}{fg=normal text.fg}
\beamertemplatenavigationsymbolsempty
\usepackage{lmodern}
\usepackage{amssymb,amsmath}
\usepackage{ifxetex,ifluatex}
\usepackage{fixltx2e} % provides \textsubscript
\ifnum 0\ifxetex 1\fi\ifluatex 1\fi=0 % if pdftex
  \usepackage[T1]{fontenc}
  \usepackage[utf8]{inputenc}
\else % if luatex or xelatex
  \ifxetex
    \usepackage{mathspec}
  \else
    \usepackage{fontspec}
  \fi
  \defaultfontfeatures{Ligatures=TeX,Scale=MatchLowercase}
\fi
\usetheme[]{CambridgeUS}
\usecolortheme{beaver}
\usefonttheme{structurebold}
% use upquote if available, for straight quotes in verbatim environments
\IfFileExists{upquote.sty}{\usepackage{upquote}}{}
% use microtype if available
\IfFileExists{microtype.sty}{%
\usepackage{microtype}
\UseMicrotypeSet[protrusion]{basicmath} % disable protrusion for tt fonts
}{}
\newif\ifbibliography
\hypersetup{
            pdftitle={A1 Erste Schritte mit R},
            pdfauthor={Jan-Philipp Kolb},
            pdfborder={0 0 0},
            breaklinks=true}
\urlstyle{same}  % don't use monospace font for urls
\usepackage{color}
\usepackage{fancyvrb}
\newcommand{\VerbBar}{|}
\newcommand{\VERB}{\Verb[commandchars=\\\{\}]}
\DefineVerbatimEnvironment{Highlighting}{Verbatim}{commandchars=\\\{\}}
% Add ',fontsize=\small' for more characters per line
\usepackage{framed}
\definecolor{shadecolor}{RGB}{248,248,248}
\newenvironment{Shaded}{\begin{snugshade}}{\end{snugshade}}
\newcommand{\KeywordTok}[1]{\textcolor[rgb]{0.13,0.29,0.53}{\textbf{#1}}}
\newcommand{\DataTypeTok}[1]{\textcolor[rgb]{0.13,0.29,0.53}{#1}}
\newcommand{\DecValTok}[1]{\textcolor[rgb]{0.00,0.00,0.81}{#1}}
\newcommand{\BaseNTok}[1]{\textcolor[rgb]{0.00,0.00,0.81}{#1}}
\newcommand{\FloatTok}[1]{\textcolor[rgb]{0.00,0.00,0.81}{#1}}
\newcommand{\ConstantTok}[1]{\textcolor[rgb]{0.00,0.00,0.00}{#1}}
\newcommand{\CharTok}[1]{\textcolor[rgb]{0.31,0.60,0.02}{#1}}
\newcommand{\SpecialCharTok}[1]{\textcolor[rgb]{0.00,0.00,0.00}{#1}}
\newcommand{\StringTok}[1]{\textcolor[rgb]{0.31,0.60,0.02}{#1}}
\newcommand{\VerbatimStringTok}[1]{\textcolor[rgb]{0.31,0.60,0.02}{#1}}
\newcommand{\SpecialStringTok}[1]{\textcolor[rgb]{0.31,0.60,0.02}{#1}}
\newcommand{\ImportTok}[1]{#1}
\newcommand{\CommentTok}[1]{\textcolor[rgb]{0.56,0.35,0.01}{\textit{#1}}}
\newcommand{\DocumentationTok}[1]{\textcolor[rgb]{0.56,0.35,0.01}{\textbf{\textit{#1}}}}
\newcommand{\AnnotationTok}[1]{\textcolor[rgb]{0.56,0.35,0.01}{\textbf{\textit{#1}}}}
\newcommand{\CommentVarTok}[1]{\textcolor[rgb]{0.56,0.35,0.01}{\textbf{\textit{#1}}}}
\newcommand{\OtherTok}[1]{\textcolor[rgb]{0.56,0.35,0.01}{#1}}
\newcommand{\FunctionTok}[1]{\textcolor[rgb]{0.00,0.00,0.00}{#1}}
\newcommand{\VariableTok}[1]{\textcolor[rgb]{0.00,0.00,0.00}{#1}}
\newcommand{\ControlFlowTok}[1]{\textcolor[rgb]{0.13,0.29,0.53}{\textbf{#1}}}
\newcommand{\OperatorTok}[1]{\textcolor[rgb]{0.81,0.36,0.00}{\textbf{#1}}}
\newcommand{\BuiltInTok}[1]{#1}
\newcommand{\ExtensionTok}[1]{#1}
\newcommand{\PreprocessorTok}[1]{\textcolor[rgb]{0.56,0.35,0.01}{\textit{#1}}}
\newcommand{\AttributeTok}[1]{\textcolor[rgb]{0.77,0.63,0.00}{#1}}
\newcommand{\RegionMarkerTok}[1]{#1}
\newcommand{\InformationTok}[1]{\textcolor[rgb]{0.56,0.35,0.01}{\textbf{\textit{#1}}}}
\newcommand{\WarningTok}[1]{\textcolor[rgb]{0.56,0.35,0.01}{\textbf{\textit{#1}}}}
\newcommand{\AlertTok}[1]{\textcolor[rgb]{0.94,0.16,0.16}{#1}}
\newcommand{\ErrorTok}[1]{\textcolor[rgb]{0.64,0.00,0.00}{\textbf{#1}}}
\newcommand{\NormalTok}[1]{#1}
\usepackage{longtable,booktabs}
\usepackage{caption}
% These lines are needed to make table captions work with longtable:
\makeatletter
\def\fnum@table{\tablename~\thetable}
\makeatother
\usepackage{graphicx,grffile}
\makeatletter
\def\maxwidth{\ifdim\Gin@nat@width>\linewidth\linewidth\else\Gin@nat@width\fi}
\def\maxheight{\ifdim\Gin@nat@height>\textheight0.8\textheight\else\Gin@nat@height\fi}
\makeatother
% Scale images if necessary, so that they will not overflow the page
% margins by default, and it is still possible to overwrite the defaults
% using explicit options in \includegraphics[width, height, ...]{}
\setkeys{Gin}{width=\maxwidth,height=\maxheight,keepaspectratio}

% Prevent slide breaks in the middle of a paragraph:
\widowpenalties 1 10000
\raggedbottom

\AtBeginPart{
  \let\insertpartnumber\relax
  \let\partname\relax
  \frame{\partpage}
}
\AtBeginSection{
  \ifbibliography
  \else
    \let\insertsectionnumber\relax
    \let\sectionname\relax
    \frame{\sectionpage}
  \fi
}
\AtBeginSubsection{
  \let\insertsubsectionnumber\relax
  \let\subsectionname\relax
  \frame{\subsectionpage}
}

\setlength{\parindent}{0pt}
\setlength{\parskip}{6pt plus 2pt minus 1pt}
\setlength{\emergencystretch}{3em}  % prevent overfull lines
\providecommand{\tightlist}{%
  \setlength{\itemsep}{0pt}\setlength{\parskip}{0pt}}
\setcounter{secnumdepth}{0}

\title{A1 Erste Schritte mit R}
\author{Jan-Philipp Kolb}
\date{15 Oktober 2018}

\begin{document}
\frame{\titlepage}

\begin{frame}{Disclaimer/ Informationen vorab}

Normalerweise gibt es große Unterschiede in den Kenntnissen und
Fähigkeiten der Teilnehmer - bitte gebt Bescheid, wenn es zu schnell
oder zu langsam geht oder etwas unklar geblieben ist.

\begin{itemize}
\tightlist
\item
  Wenn es Fragen gibt - immer fragen
\item
  In diesem Kurs gibt es viele
  \href{http://web.math.ku.dk/~helle/R-intro/exercises.pdf}{\textbf{Übungen}},
  denn das Programmieren lernt man am Ende nur allein.
\item
  Ich habe viele \href{https://www.showmeshiny.com/}{\textbf{Beispiele}}
  - probiert sie aus
\item
  R macht mehr Spaß zusammen - arbeitet zusammen!
\end{itemize}

\end{frame}

\begin{frame}{Gründe R zu nutzen\ldots{}}

\begin{itemize}
\item
  \ldots{} R ist eine
  \href{https://stackoverflow.com/questions/1546583/what-is-the-definition-of-an-open-source-programming-language}{\textbf{quelloffene
  Sprache}}
\item
  \ldots{} hervorragende
  \href{http://matthewlincoln.net/2014/12/20/adjacency-matrix-plots-with-r-and-ggplot2.html}{\textbf{Grafiken}},
  \href{https://www.r-bloggers.com/3d-plots-with-ggplot2-and-plotly\%20/}{\textbf{Grafiken}},
  \href{https://procomun.wordpress.com/2011/03/18/splomr/}{\textbf{Grafiken}}
\item
  \ldots{} \href{https://github.com/Japhilko/RInterfaces}{\textbf{R kann
  in Kombination mit anderen Programmen verwendet werden}} - z.B. zur
  \href{https://github.com/Japhilko/RInterfaces/blob/master/slides/Datenimport.md}{\textbf{Verknüpfung
  von Daten}}
\item
  \ldots{} R kann
  \href{https://cran.r-project.org/web/packages/MplusAutomation/index.html}{\textbf{zur
  Automatisierung}} verwendet werden
\item
  \ldots{} Breite und aktive Community -
  \href{https://www.r-bloggers.com/}{\textbf{Man kann die Intelligenz
  anderer Leute nutzen ;-)}}
\end{itemize}

\end{frame}

\begin{frame}{R kann in Kombination mit anderen Programmen genutzt
werden\ldots{}}

\includegraphics{figure/Rinterfaces.PNG}

\begin{itemize}
\tightlist
\item
  Schnittstelle zu:
  \href{https://cran.r-project.org/web/packages/reticulate/vignettes/calling_python.html}{\textbf{Python}},
  \href{https://www.springer.com/de/book/9781441900517}{\textbf{Excel}},
  \href{https://www.ibm.com/support/knowledgecenter/en/SSFUEU_7.2.0/com.ibm.swg.ba.cognos.op_capmod_ig.7.2.0.doc/t_essentials_for_r_statistics.html}{\textbf{SPSS}},
  \href{https://cran.r-project.org/web/packages/SASmixed/index.html}{\textbf{SAS}},
  \href{https://cran.r-project.org/web/packages/RStata/index.html}{\textbf{Stata}}
\end{itemize}

\end{frame}

\begin{frame}{R für SPSS Nutzer}

\begin{block}{Bob Muenchen -
\href{https://science.nature.nps.gov/im/datamgmt/statistics/r/documents/r_for_sas_spss_users.pdf}{\textbf{R
for SPSS and SAS Users }}}

\begin{itemize}
\tightlist
\item
  \href{http://www.rcommander.com/}{\textbf{R commander (Rcmdr)}}
\end{itemize}

\includegraphics{figure/Rcommanderex.PNG}

\end{block}

\end{frame}

\begin{frame}{\href{https://www.bloomberg.com/news/articles/2013-04-18/faq-reinhart-rogoff-and-the-excel-error-that-changed-history}{R
sollte genutzt werden, weil andere Programme Fehler provozieren:}}

\includegraphics{figure/RheinhartRogoff.PNG}

\end{frame}

\begin{frame}{\href{https://gallery.shinyapps.io/cran-gauge/}{\textbf{Die
Beliebtheit von R-Paketen}}}

\includegraphics{figure/CRANdownloads.PNG}

\end{frame}

\begin{frame}{Download R:}

\url{http://www.r-project.org/}

\includegraphics{figure/CRAN1picture.PNG}

\end{frame}

\begin{frame}{Open Source Programm R}

\begin{block}{Das ist das Basis-R:}

\includegraphics{figure/BasisR.PNG}

\end{block}

\end{frame}

\begin{frame}{Graphical user interface}

Viele Leute nutzen ein
\href{https://en.wikipedia.org/wiki/Graphical_user_interface}{\textbf{Graphical
User Interface}} (GUI) oder ein
\href{https://en.wikipedia.org/wiki/Integrated_development_environment}{\textbf{Integrated
Development Interface}} (IDE).

Aus den folgenden Gründen:

\begin{itemize}
\tightlist
\item
  Syntax-Hervorhebung
\item
  Auto-Vervollständigung
\item
  Bessere Übersicht über Graphiken, Pakete, Dateien, \ldots{}
\end{itemize}

\end{frame}

\begin{frame}{Various text editors / IDEs}

\begin{itemize}
\item
  \href{https://projects.gnome.org/gedit/}{\textbf{Gedit}} with
  R-specific Add-ons for Linux
\item
  \href{http://www.gnu.org/software/emacs/}{\textbf{Emacs}} and ESS
  (Emacs speaks statistics)- An extensible, customizable, free/libre
  text editor --- and more.
\item
  I use \href{https://www.rstudio.com/}{\textbf{Rstudio!}}
\end{itemize}

\includegraphics{figure/0_overall.jpg}

\end{frame}

\begin{frame}{RStudio}

\includegraphics{figure/RstudioExample.PNG}

\end{frame}

\begin{frame}{Customizing RStudio}

\begin{block}{Six
\href{http://www.r-bloggers.com/top-6-reasons-you-need-to-be-using-rstudio/}{\textbf{reasons}}
to use
\href{https://support.rstudio.com/hc/en-us/articles/200549016-Customizing-RStudio}{\textbf{Rstudio}}.}

\includegraphics{figure/options_general.png}

\end{block}

\end{frame}

\begin{frame}[fragile]{A1A Exercise - Preparation}

\begin{itemize}
\tightlist
\item
  Check if R is installed on your computer.
\item
  If not, download \href{r-project.org}{\textbf{R}} and install it.
\item
  Check if Rstudio is installed.
\item
  If not - \href{http://www.rstudio.com/}{\textbf{install}} Rstudio.
\item
  Start RStudio. Go to the console (lower left window) and write
\end{itemize}

\begin{Shaded}
\begin{Highlighting}[]
\DecValTok{3}\OperatorTok{+}\DecValTok{2}
\end{Highlighting}
\end{Shaded}

\begin{itemize}
\tightlist
\item
  If there is not already an editor open in the upper left window, then
  go to the file menu and open a new script. Check the date with
  \texttt{date()} and the R version with \texttt{sessionInfo()}.
\end{itemize}

\begin{Shaded}
\begin{Highlighting}[]
\KeywordTok{date}\NormalTok{()}
\end{Highlighting}
\end{Shaded}

\begin{Shaded}
\begin{Highlighting}[]
\KeywordTok{sessionInfo}\NormalTok{()}
\end{Highlighting}
\end{Shaded}

\end{frame}

\begin{frame}[fragile]{R ist eine objektorientierte Sprache.}

\begin{block}{Vektoren und Zuweisungen}

\begin{itemize}
\tightlist
\item
  \texttt{\textless{}-} ist der Zuweisungsoperator
\end{itemize}

\begin{Shaded}
\begin{Highlighting}[]
\NormalTok{b <-}\StringTok{ }\KeywordTok{c}\NormalTok{(}\DecValTok{1}\NormalTok{,}\DecValTok{2}\NormalTok{) }\CommentTok{# create an object with the numbers 1 and 2}
\end{Highlighting}
\end{Shaded}

\begin{itemize}
\tightlist
\item
  Auf dieses Objekt kann eine Funktion angewendet werden:
\end{itemize}

\begin{Shaded}
\begin{Highlighting}[]
\KeywordTok{mean}\NormalTok{(b) }\CommentTok{# computes the mean}
\end{Highlighting}
\end{Shaded}

\begin{verbatim}
## [1] 1.5
\end{verbatim}

Mit den folgenden Funktionen können wir etwas über die Eigenschaften des
Objekts erfahren:

\begin{Shaded}
\begin{Highlighting}[]
\KeywordTok{length}\NormalTok{(b) }\CommentTok{# b has the length 2}
\end{Highlighting}
\end{Shaded}

\begin{verbatim}
## [1] 2
\end{verbatim}

\end{block}

\begin{block}{Objektstruktur}

\begin{Shaded}
\begin{Highlighting}[]
\KeywordTok{str}\NormalTok{(b) }\CommentTok{# b is a numeric vector}
\end{Highlighting}
\end{Shaded}

\begin{verbatim}
##  num [1:2] 1 2
\end{verbatim}

\end{block}

\end{frame}

\begin{frame}{Funktionen in base-Paket}

\begin{longtable}[]{@{}lll@{}}
\toprule
Function & Meaning & Example\tabularnewline
\midrule
\endhead
str() & Object structure & str(b)\tabularnewline
max() & Maximum & max(b)\tabularnewline
min() & Minimum & min(b)\tabularnewline
sd() & Standard deviation & sd(b)\tabularnewline
var() & Variance & var(b)\tabularnewline
mean() & Mean & mean(b)\tabularnewline
median() & Median & median(b)\tabularnewline
\bottomrule
\end{longtable}

Diese Funktionen benötigen nur ein Argument.

\end{frame}

\begin{frame}[fragile]{Funktionen mit mehr Argumenten}

\begin{block}{Andere Funktionen benötigen mehr Argumente:}

\begin{longtable}[]{@{}lll@{}}
\toprule
Argument & Bedeutung & Beispiel\tabularnewline
\midrule
\endhead
quantile() & 90 \% Quantile & quantile(b,.9)\tabularnewline
sample() & Draw a sample & sample(b,1)\tabularnewline
\bottomrule
\end{longtable}

\begin{Shaded}
\begin{Highlighting}[]
\KeywordTok{quantile}\NormalTok{(b,.}\DecValTok{9}\NormalTok{)}
\end{Highlighting}
\end{Shaded}

\begin{verbatim}
## 90% 
## 1.9
\end{verbatim}

\begin{Shaded}
\begin{Highlighting}[]
\KeywordTok{sample}\NormalTok{(b,}\DecValTok{1}\NormalTok{) }
\end{Highlighting}
\end{Shaded}

\begin{verbatim}
## [1] 1
\end{verbatim}

\end{block}

\begin{block}{Beispiele - Funktionen mit mehr als einem Argument}

\begin{Shaded}
\begin{Highlighting}[]
\KeywordTok{max}\NormalTok{(b)}
\end{Highlighting}
\end{Shaded}

\begin{verbatim}
## [1] 2
\end{verbatim}

\begin{Shaded}
\begin{Highlighting}[]
\KeywordTok{min}\NormalTok{(b)}
\end{Highlighting}
\end{Shaded}

\begin{verbatim}
## [1] 1
\end{verbatim}

\begin{Shaded}
\begin{Highlighting}[]
\KeywordTok{sd}\NormalTok{(b)}
\end{Highlighting}
\end{Shaded}

\begin{verbatim}
## [1] 0.7071068
\end{verbatim}

\begin{Shaded}
\begin{Highlighting}[]
\KeywordTok{var}\NormalTok{(b)}
\end{Highlighting}
\end{Shaded}

\begin{verbatim}
## [1] 0.5
\end{verbatim}

\end{block}

\begin{block}{Funktionen mit einem Argument}

\begin{Shaded}
\begin{Highlighting}[]
\KeywordTok{mean}\NormalTok{(b)}
\end{Highlighting}
\end{Shaded}

\begin{verbatim}
## [1] 1.5
\end{verbatim}

\begin{Shaded}
\begin{Highlighting}[]
\KeywordTok{median}\NormalTok{(b)}
\end{Highlighting}
\end{Shaded}

\begin{verbatim}
## [1] 1.5
\end{verbatim}

\end{block}

\end{frame}

\begin{frame}{\href{http://cran.r-project.org/doc/manuals/R-intro.html}{\textbf{Überblick
Funktionen}}}

\url{http://cran.r-project.org/doc/manuals/R-intro.html}

\includegraphics{figure/UebersichtBefehle.PNG}

\end{frame}

\begin{frame}[fragile]{A1B Übung - Zuweisungen und Funktionen}

Erstellen Sie einen Vektor \texttt{b} mit den Zahlen von 1 bis 5 und
berechnen Sie\ldots{}.

\begin{enumerate}
\def\labelenumi{\arabic{enumi}.}
\item
  den Mittelwert
\item
  die Varianz
\item
  die Standardabweichung
\item
  die Quadratwurzel aus dem Mittelwert
\end{enumerate}

\end{frame}

\begin{frame}{\href{https://stats.idre.ucla.edu/r/seminars/intro/}{\textbf{Wo
man Routinen findet}}}

\begin{itemize}
\tightlist
\item
  Viele Funktionen sind in Basis-R enthalten.
\item
  Viele spezifische Funktionen sind in zusätzliche Bibliotheken
  integriert.
\item
  R kann modular durch sogenannte Pakete oder Bibliotheken erweitert
  werden.
\item
  Die wichtigsten Pakete, die auf CRAN gehostet werden (13087 at Do Okt
  04)
\item
  Weitere Pakete finden Sie z.B. unter
  \href{www.bioconductor.org}{\textbf{Bioleiter}}
\end{itemize}

\begin{block}{Übersicht R-Pakete}

\includegraphics{figure/Packages.PNG}

\end{block}

\end{frame}

\begin{frame}[fragile]{Installation von Paketen}

\begin{itemize}
\tightlist
\item
  Die Anführungszeichen um den Paketnamen herum sind für den Befehl
  \texttt{install.packages} notwendig.
\item
  Sie sind optional für den Befehl \texttt{library}.
\item
  Man kann auch \texttt{require} anstelle von \texttt{library}
  verwenden.
\end{itemize}

\begin{Shaded}
\begin{Highlighting}[]
\KeywordTok{install.packages}\NormalTok{(}\StringTok{"lme4"}\NormalTok{)}

\KeywordTok{library}\NormalTok{(lme4)}
\end{Highlighting}
\end{Shaded}

\end{frame}

\begin{frame}{Installation von Paketen mit RStudio}

\includegraphics{figure/PaketeRstudio.PNG}

\end{frame}

\begin{frame}{Bestehende Pakete und Installation}

\includegraphics{figure/packages3.PNG}

\end{frame}

\begin{frame}[fragile]{Übersicht über viele nützliche Pakete:}

\begin{itemize}
\tightlist
\item
  Luhmann -
  \href{http://www.beltz.de/fileadmin/beltz/downloads/OnlinematerialienPVU/28090_Luhmann/Verwendete\%20Pakete.pdf}{\textbf{Table
  with many useful packages}}
\end{itemize}

\begin{block}{Weitere interessante Pakete:}

\begin{itemize}
\item
  Paket für Import/Export -
  \href{http://cran.r-project.org/web/packages/foreign/foreign.pdf}{\textbf{\texttt{foreign}}}
\item
  \href{http://iase-web.org/documents/papers/icots8/ICOTS8_4J1_TILLE.pdf}{\textbf{\texttt{sampling}-Paket
  für die Stichprobenziehung}}
\item
  \texttt{xtable} Paket zur Integration von LateX in R
  (\href{http://cran.r-project.org/web/packages/xtable/vignettes/xtableGallery.pdf}{\textbf{xtable
  Galerie}})
\item
  \href{http://cran.r-project.org/web/packages/dummies/dummies.pdf}{\textbf{\texttt{dummies}
  - Paket zur Erstellung von Dummies}}
\item
  \href{http://cran.r-project.org/web/packages/mvtnorm/index.html}{\textbf{Paket
  \texttt{mvtnorm} um eine multivariate Normalverteilung zu erhalten. }}
\item
  \href{http://www.r-bloggers.com/tag/maptools/}{\textbf{Paket
  \texttt{maptools} um Karten zu erzeugen}}
\end{itemize}

\end{block}

\end{frame}

\begin{frame}[fragile]{Pakete aus verschiedenen Quellen installieren}

\begin{block}{Pakete vom CRAN Server installieren}

\begin{Shaded}
\begin{Highlighting}[]
\KeywordTok{install.packages}\NormalTok{(}\StringTok{"lme4"}\NormalTok{)}
\end{Highlighting}
\end{Shaded}

\end{block}

\begin{block}{Pakete vom Bioconductor Server installieren}

\begin{Shaded}
\begin{Highlighting}[]
\KeywordTok{source}\NormalTok{(}\StringTok{"https://bioconductor.org/biocLite.R"}\NormalTok{)}
\KeywordTok{biocLite}\NormalTok{(}\KeywordTok{c}\NormalTok{(}\StringTok{"GenomicFeatures"}\NormalTok{, }\StringTok{"AnnotationDbi"}\NormalTok{))}
\end{Highlighting}
\end{Shaded}

\end{block}

\begin{block}{Pakete von Github installieren}

\begin{Shaded}
\begin{Highlighting}[]
\KeywordTok{install.packages}\NormalTok{(}\StringTok{"devtools"}\NormalTok{)}
\KeywordTok{library}\NormalTok{(devtools)}

\KeywordTok{install_github}\NormalTok{(}\StringTok{"hadley/ggplot2"}\NormalTok{)}
\end{Highlighting}
\end{Shaded}

\end{block}

\end{frame}

\begin{frame}{Wie bekomme ich einen Überblick?}

\begin{itemize}
\item
  Entdecke Pakete, die kürzlich auf den
  \href{https://mran.microsoft.com/packages/}{\textbf{CRAN}} Server
  hochgeladen wurden
\item
  Nutze eine Shiny Web-App, die
  \href{https://gallery.shinyapps.io/cran-gauge/}{\textbf{Pakete
  anzeigt, die kürzlich von CRAN}} heruntergeladen wurden.
\item
  Werfe einen Blick auf eine
  \href{https://support.rstudio.com/hc/en-us/articles/201057987-Quick-list-of-useful-R-packages}{\textbf{Quick-Liste
  nützlicher Pakete}}
\item
  \ldots{}., oder auf eine Liste mit den
  \href{http://www.computerworld.com/article/2921176/business-intelligence/great-r-packages-for-data-import-wrangling-visualization.html}{\textbf{besten
  Paketen für die Datenverarbeitung und -analyse}},\ldots{}..
\item
  \ldots{}., oder schaue unter
  \href{https://www.r-bloggers.com/the-50-most-used-r-packages/}{\textbf{die
  50 meistgenutzten Pakete}}
\end{itemize}

\end{frame}

\begin{frame}[fragile]{CRAN Task Views}

\begin{itemize}
\tightlist
\item
  Bezüglich mancher Themen gibt es einen Überblick über alle wichtigen
  Pakete - (\href{https://cran.r-project.org/web/views/}{\textbf{CRAN
  Task Views}})
\item
  Momentan gibt es 35 Task Views.
\item
  Alle Pakete einer Task-View können mit folgendem Befehl installiert
  werden:
  \href{https://mran.microsoft.com/rpackages/}{\textbf{command:}}
\end{itemize}

\begin{Shaded}
\begin{Highlighting}[]
\KeywordTok{install.packages}\NormalTok{(}\StringTok{"ctv"}\NormalTok{)}
\KeywordTok{library}\NormalTok{(}\StringTok{"ctv"}\NormalTok{)}
\KeywordTok{install.views}\NormalTok{(}\StringTok{"Bayesian"}\NormalTok{)}
\end{Highlighting}
\end{Shaded}

\includegraphics{figure/CRANtaskViews.PNG}

\end{frame}

\begin{frame}{A1C Übung - zusätzliche Pakete}

Geht auf \url{https://cran.r-project.org/} und sucht nach
Paketen\ldots{}

\begin{itemize}
\tightlist
\item
  die sich für die deskriptive Datenanalyse eignen.
\item
  mit denen man fremde Datensätze einlesen kann (z.B. SPSS data)
\item
  mit denen man Lasso Regressionen rechnen kann
\item
  mit denen man große Datenmengen bearbeiten kann
\end{itemize}

\end{frame}

\begin{frame}{Links zum Weiterlesen:}

\begin{itemize}
\item
  \href{http://www.r-bloggers.com/why-you-should-learn-r-first-for-data-science/}{\textbf{Warum
  man R zuerst lerneen sollte wenn man Data Science machen möchte}}
\item
  RStudio hat den
  \href{http://www.r-bloggers.com/rstudio-infoworld-2015-technology-of-the-year-award-recipient/}{\textbf{Infoworld
  2015 Technology of the Year Award.}} bekommen
\item
  \href{http://www.fastcolabs.com/3030063/why\%20the\%20r\%20programming\%20language\%20is\%20good\%20for\%20business}{\textbf{Warum
  R gut für Unternehmen ist}}
\item
  Schaut auf
  \href{http://www.r-bloggers.com/why-use-r/}{\textbf{R-bloggers}} 
\item
  Vergleich zwischen
  \href{http://www.dataschool.io/python-or-r-for-data-science/}{\textbf{python
  und R}}
\item
  R und Stata -
  \href{http://economistry.com/2013/11/r-impact-evaluation-r-stata-side-side/}{\textbf{Side-by-side}}
\item
  \href{https://awesome-r.com/}{\textbf{AWESOME R}}
\item
  \href{https://support.bioconductor.org/p/33781/}{\textbf{1000 R
  tutorials/Links}}
\item
  \href{https://www.youtube.com/playlist?list=PLcgz5kNZFCkzSyBG3H-rUaPHoBXgijHfC}{\textbf{Zwei
  Minuten Videos auf Github}}
\end{itemize}

\end{frame}

\begin{frame}{Shiny App - Einführung in R}

\url{http://www.intro-stats.com/}

\includegraphics{figure/intror_shiny.PNG}

\end{frame}

\end{document}
